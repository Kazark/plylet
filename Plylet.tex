\documentclass{article}

\usepackage{amssymb}
\usepackage{amsmath}

\begin{document}

The unit type is built in. Because this is not dependently typed, we can place
values and types in two different namespaces, so let's follow the convention of
using empty tuple syntax $()$ at both the value and type level.

\[ \frac{}{\Gamma\vdash()\mathbin{:}()}\quad\text{(Unit introduction)} \]

Implicit coercion ($\leadsto$) is, after type-of ($:$), a second kind of
judgment. Coercions are useful on input types of functions; implicit coercions
are supported via applications in such a first-class manner that we actually
bake it into the application rule.

\[ \frac{\Gamma\vdash f\mathbin{:}\alpha\to\beta\quad\Gamma\vdash x\mathbin{:}\gamma\quad\Gamma\vdash \gamma\leadsto\alpha}{\Gamma\vdash f~x\mathbin{:}\beta}\quad\text{(Function application)} \]

The obvious and immediate difficulty is, if you have
$f\mathbin{:}\alpha\to\beta$ and $x\mathbin{:}\alpha$, how do you apply that?
Whereas the purpose of coercions is to give us a flexible framework for typesafe
function application, we want to make this base case always possible, so we
introduce a rule for coercion reflexivity. However we don't want to introduce
any complex system for saying ``we have a type in the environment'', so instead
we define the rule in this form, which is simple (in terms of not requiring any
more constructs, such as parametric polymorphism, to be introduced into the
system) and useful (in that it is immediately applicable at the point that you
actually need it).

\[ \frac{\Gamma\vdash x\mathbin{:}\alpha}{\Gamma\vdash \alpha\leadsto\alpha}\quad\text{(Coercion reflexivity)} \]

We also want $\textbf{let}\dots\textbf{in}\dots$ expressions.

\[ \frac{\Gamma\vdash M\mathbin{:}\alpha\quad\Gamma\vdash P[x:=M]\mathbin{:}\beta}{\Gamma\vdash \textbf{let}\, x = M \,\textbf{in}\, P\mathbin{:}\beta}\quad\text{(Let expression)} \]

We also want to be able to construct pairs. We do not need to be able to
deconstruct them within the scope of this theory, so we only add an introduction
rule.

\[ \frac{\Gamma\vdash x\mathbin{:}\alpha\quad\Gamma\vdash y\mathbin{:}\beta}{\Gamma\vdash (x, y)\mathbin{:}\alpha*\beta}\quad\text{(Pair introduction)} \]

Thus, only two type constructors are supported out of the box, both binary:
$\to$ for function types, and $*$ for pair types.

It is also desirable to be allowed to sequence expressions, where the first one
is of type unit---in which case, we can think of it as a statement---on separate
lines, which we denote here for the sake of the syntax with the standard C-style
line separator $;$.

\[ \frac{\Gamma\vdash x\mathbin{:}()\quad\Gamma\vdash y\mathbin{:}\alpha}{\Gamma\vdash (x\mathbin{;}y)\mathbin{:}\alpha}\quad\text{(Expression combination)} \]

This would allow us to write blocks of imperative code, as in ML dialects. (The
interpreter will want to consider Plylet to be an ``inside-the-monad''
perspective for whatever monad it interprets it within).

\end{document}

