\documentclass{article}

\usepackage{amssymb}
\usepackage{amsmath}

\begin{document}

The unit type is built in. Because this is not dependently typed, we can place
values and types in two different namespaces. Moreover, we don't even need to
let the value of unit show up in the surface language. So let's follow the
convention of using empty tuple syntax $()$ at both the value and type level.

\[ \frac{}{\Gamma\vdash()\mathbin{:}()}\quad\text{(Unit)} \]

Implicit coercion ($\leadsto$) is, after type-of ($:$), a second kind of
judgment. Coercions are useful on input types of functions; implicit coercions
are supported via applications in such a first-class manner that we actually
bake it into the application rule.

\[ \frac{\Gamma\vdash f\mathbin{:}\alpha\to\beta\quad\Gamma\vdash x\mathbin{:}\gamma\quad\Gamma\vdash \gamma\leadsto\alpha}{\Gamma\vdash f~x\mathbin{:}\beta}\quad\text{(Function application)} \]

The obvious and immediate difficulty is, if you have
$f\mathbin{:}\alpha\to\beta$ and $x\mathbin{:}\alpha$, how do you apply that?
Whereas the purpose of coercions is to give us a flexible framework for typesafe
function application, we want to make this base case always possible, so we
introduce a rule for coercion reflexivity. However we don't want to introduce
any complex system for saying ``we have a type in the environment'', so instead
we define the rule in this form, which is simple (in terms of not requiring any
more constructs to be introduced into the system) and useful (in that it is
immediately applicable at the point that you actually need it).

\[ \frac{\Gamma\vdash x\mathbin{:}\alpha}{\Gamma\vdash \alpha\leadsto\alpha}\quad\text{(Coercion reflexivity)} \]

We also want $\textbf{let}\dots\textbf{in}\dots$ expressions.

\[ \frac{\Gamma\vdash M\mathbin{:}\alpha\quad\Gamma\vdash
    P[x:=M]\mathbin{:}\beta}{\Gamma\vdash \textbf{let}\, x = M \,\textbf{in}\, P\mathbin{:}\beta}\quad\text{(Let expression)} \]

%\[ \frac{c \text{ is a constant of type } T}{\Gamma\vdash c\mathbin{:}T} \]
%\[ \frac{\Gamma,x\mathbin{:}\sigma\vdash e\mathbin{:}\tau}{\Gamma\vdash (\lambda x\mathbin{:}\sigma.~e)\mathbin{:}(\sigma \to \tau)} \]
%\[ \frac{\Gamma\vdash e_1\mathbin{:}\sigma\to\tau\quad\Gamma\vdash e_2\mathbin{:}\sigma}{\Gamma\vdash e_1~e_2\mathbin{:}\tau} \]

\end{document}

